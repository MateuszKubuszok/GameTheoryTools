\documentclass[polish]{standalone}
\usepackage{thesis}

\begin{document}
\pagestyle{headings}

\chapter{Implementacja}

\section{Wprowadzenie}

W rozdziale tym zostaną opisane szczegóły implementacji projektu. Uwzględnimy zarówno wykorzystane narzędzia, sposób
organizacji kodu jak i informacje na temat zastosowanych rozwiązań. Nie będziemy się koncentrować na przeglądzie kodu
źródłowego, ale raczej na wykorzystanych algorytmach i strukturach danych. Na koniec omówimy praktyki przyjęte, aby
umożliwić stworzenie wysokiej jakości kodu.

\subsection{Użyte narzędzia}

Biblioteka \LIB{GLPK} wykorzystana w tym projekcie napisana została w języku \LANG{C}. Z tego powodu język, w których
docelowo miał zostać zrealizowany projekt powinien był udostępniać możliwość wykorzystywania bibliotek języka \LANG{C}
za pośrednictwem adapterów, bądź bezpośrednio. Po uwzględnieniu narzędzi istniejących dla poszczególnych języków
zdecydowano się na wybór języka \LANG{C++} - adaptacja bibliotek z \LANG{C} jest naturalna, a przy tym sam język
udostępnia możliwość programowania obiektowego, jak i dużą elastyczność przy zachowaniu szybkości działania.

Kolejnym istotnym argumentem za \LANG{C++} była dostępność bibliotek i pakietów do tworzenia parserów gramatyk
bezkontekstowych - w projekcie do stworzenia parsera wczytującego dane wykorzystane zostały \PROG{flex} oraz
\PROG{bison}. Umożliwiają one utworzenie bardzo wydajnego parsera \CODE{LALR(1)}, zdolnego do uruchamiania
zdefiniowanego kodu \LANG{C}/\LANG{C++}.

Projekt wykorzystuje również bibliotekę \LIB{Boost} - z wyjątkiem kilku miejsc \LIB{Boost.Container} całkowicie
zastąpił w nim kontenery biblioteki \LIB{STL}, pozostałe komponenty biblioteki \LIB{Boost} zostały użyte jako
uzupełnienie biblioteki \LIB{STL}.

\subsection{Wewnętrzna struktura projektu}

Niemal wszystkie komponenty umieszczone zostały we wspólnej przestrzeni nazw \CODE{GT}, przy czym klasy pogrupowano
w kolejne 4 podprzestrzenie nazw:
\begin{itemize}
\item \CODE{Model} - klasy implementujące model danych bądź powiązane,
\item \CODE{Routines} - procedury obliczające równowagi Nasha dla poszczególnych rodzajów gier,
\item \CODE{GTL} - klasy parsera danych wejściowych,
\item \CODE{Program} - klasy powiązane z utworzeniem pliku wykonywalnego parsera.
\end{itemize}
Wyjątek stanowiły jedynie elementy, które do poprawnego działania, wymagały umieszczenia w przestrzeni nazw \CODE{std},
jak np. niektóre przeciążenia operatorów.

\section{Model danych}

Ze względu na odmienne założenia odnośnie danych dla gier w formie strategicznej oraz ekstensywnej, dla każdego rodzaju
gry zdefiniowano osobny model danych (jednak tam, gdzie to możliwe, uwspólniono ich interfejs).

Przyjęto założenie, że raz utworzonego modelu danych nie modyfikuje się - z tego powodu wszystkie gotowe instancje
przechowuje się przy pomocy inteligentnych wskaźników na stałe obiekty - zmniejsza to narzut związany z kopiowaniem
obiektów, a także umożliwia zastosowanie co bardziej kosztownych operacji, jako jednorazowych kosztów występujących
w chwili utworzenia obiektu.

\subsection{Gracz}

Wspólnym elementem interfejsu są przede wszystkim instancje graczy. Każda z nich przechowuje informację na temat nazwy
identyfikującej gracza, a także tablicę zawierającą strategie oraz mapę indeksów jego strategii - dzięki temu
znalezienie nazwy na podstawie jej numeru jest możliwe w czasie $O(1)$, zaś numeru indeksu na podstawie nazwy w czasie 
$O(log\;\ABS{S})$ (gdzie $\ABS{S}$ to liczba strategii gracza) - obie te informacje są bardzo często wykorzystywane,
więc szybkie ich przeliczanie warte jest jednorazowego narzutu związanego z wypełnieniem dwóch osobnych struktur danych.

\subsection{Wartości funkcji wypłaty dla podanych strategii}

Wartości wypłaty wszystkich graczy dla danego profilu strategii czystych zamknięto w osobnym obiekcie danych - jest on
kolejnym wspólnym elementem interfejsu. Obiekt ten zwraca wypłatę dla danego gracza - wewnętrzna reprezentacja danych
zrealizowana jest przy pomocy mapy, aby zminimalizować czas dostępu do żądanej wartości ($O(log\;n)$ dla $n$ graczy).

\subsection{Gry w formie strategicznej}

Ponieważ nie przyjęto odgórnego założenia na temat liczby graczy oraz ich strategi czystych, nie można było stworzyć
modelu danych opartego na wielowymiarowej tablicy danych. Zamiast tego zdecydowano się wewnętrznie przechowywać dane
w tablicy jednowymiarowej przy pomocy funkcji mieszającej\INENG{hash function}.

Oznaczmy przez $\ABS{S_i}$ liczbę strategii gracza $i$ (gdzie $i = 1, 2, ..., n$).

Oznaczmy przez $H_1 = 1$ oraz $H_i = H_{i-1} * \ABS{S_i}$.

Oznaczmy przez $s \in S$ krotkę zawierającą po 1 strategii czystej każdego gracza, zaś $s_i$ ową strategię
dla gracza $i$.

Przypiszmy wszystkim strategiom $s_i$ gracza $i$ unikalne liczby porządkowe $O_i(s_i) = 0, 1, ..., \ABS{S_i}-1$.

Wówczas funkcję mieszającą można zdefiniować jako:
$$h(s) = \sum\limits_{i=0}^{n-1} H_i O_i(s_i)$$

\begin{theorem}
Tak zdefiniowana funkcja mieszająca jest bijekcją ze zbioru $S$ w $ \ZZ \cap [0, \prod\limits_{i=1}^{n} \ABS{S_i}] $.
\end{theorem}

\begin{proof}
Dla $n = 1$ jest to oczywiste. Dla $n > 1$ wystarczy zauważyć, że z definicji jeśli tylko byłaby to bijekcja dla $n-1$
to na pewno będzie nią dla $n$.
\end{proof}

W ogólności funkcje mieszające nie są wykorzystywane w ten sposób i wymaga się od nich zazwyczaj szybkości oraz
jednokierunkowości. Tutaj jednak używane są do indeksowania tablicy asocjacyjnej i ich odwracalność jest przydatną
właściwością - pozwala łatwo odzyskać profil na podstawie przydzielonego mu indeksu tablicy.

\subsection{Gry w formie ekstensywnej}

Gry w postaci ekstensywnej opisują struktury w formie drzewa i tak też są implementowane w programie. Każdemu 
wierzchołkowi w drzewie opowiada obiekt wierzchołka przechowujący mapę potomków - rozwiązane uznano za bardziej
naturalne (ze względu na samą naturę reprezentowanego obiektu) oraz mniej problematyczne niż tablica mieszająca,
która bez modyfikacji zakładałaby możliwość wyboru każdej z zadeklarowanych przez gracza strategii. Ponieważ dopuszczamy
możliwość wyboru dowolnego co najmniej 2-elementowego podzbioru strategii jako zbioru możliwych wyborów w danym
wierzchołku, byłby to jedynie dodatkowy problem do rozwiązania.

Przy takiej definicji modelu funkcja wypłaty przeszukuje jedynie drzewo i zwraca wartości przechowywane
w odnalezionym węźle.

\section{Procedury obliczające równowagi}

\subsection{Gry w formie strategicznej}

Ponieważ nie jest znany żaden wydajny algorytm odnajdywania równowagi w strategiach czystych w grach strategicznych
konieczne jest przeszukanie całej przestrzeni rozwiązań - testowania każdego profilu strategii jako potencjalnej
równowagi Nasha. Jedyną możliwą do zastosowania optymalizacją jest eliminacja strategii zdominowanych. Nie gwarantuje
ona jednak, że wyeliminowana zostanie jakaś znacząca ilość strategii (w najgorszym przypadku żadna).

Program korzysta z odwracalności przyjętej funkcji mieszającej, aby iterować po wszystkich jej wartościach, odzyskiwać
z nich profil strategii i testować czy jest równowagą Nasha.

Strategie mieszane w grach jedno- i dwumacierzowych jako wartości obliczane przy pomocy algorytmu simpleks, są
otrzymywane przez wypełnienie modeli biblioteki \LIB{GLPK} i oddelegowanie obliczenia wartości problemu
liniowego (lub całkowitoliczbowego) na jej procedury.

\subsubsection{Gry jednomacierzowe}

Korzystając ze wzorów \ref{MTRX2a} oraz \ref{MTRX2b} możemy zapisać problem w języku \LANG{MathProg}. Wygląda on
wówczas następująco:

\begin{mathprog}[caption=Przykładowe dane]
data;

/* Definicje graczy */
set P1S := a b c;
set P2S := 1 2 3 4;

/* Macierz wypłat A */
param Payoff
    :  1  2  3  4 :=
    a -5  3  1  8
    b  5  5  4  6
    c -4  6  0  5;
 
end;
\end{mathprog}

\begin{mathprog}[caption=Rozwiązanie dla gracza pierwszego]
set P1S;
set P2S;

param Payoff{P1S, P2S};

var y{P1S} >= 0;

/* Funkcja celu z definicji problemu dla P1 */
minimize GameValue: sum{i in P1S} y[i];

/* Ograniczenie z definicji problemu dla P1 */
s.t. Condition{j in P2S}:
    sum{i in P1S} Payoff[i,j] * y[i] >= 1;

solve;

/* Wyświetlenie wyników */
printf "Value: %s\n", GameValue;
printf "Player 1 strategies:\n";
for{i in P1S}
    printf "Found %s, actual %s\n", y[i], y[i]/GameValue; 

end;
\end{mathprog}

\begin{mathprog}[caption=Rozwiązanie dla gracza drugiego]
set P1S;
set P2S;

param Payoff{P1S, P2S};

var x{P2S} >= 0;

/* Funkcja celu z definicji problemu dla P2 */
maximize GameValue: sum{j in P2S} x[j];

/* Ograniczenie z definicji problemu dla P2 */
s.t. Condition{i in P1S}:
    sum{j in P2S} Payoff[i,j] * x[j] <= 1;

solve;

/* Wyświetlenie wyników */
printf "Value: %s\n", GameValue;
printf "Player 2 strategies:\n";
for{j in P2S}
    printf "Found %s, actual %s\n", x[j], x[j]/GameValue;
 
end;
\end{mathprog}

\subsubsection{Gra dwumacierzowa}

Do obliczania wartości gry dwumacierzowej wykorzystujemy algorytm SGC. Ograniczenia \ref{SGC2a}, \ref{SGC2b},
\ref{SGC2c}, \ref{SGC2d}, \ref{SGC2e} oraz \ref{SGC2f} wraz z definicjami zmiennych w języku \LANG{MathProg}, wyglądają
następująco:

\begin{mathprog}[caption=Przykładowe dane]
data;

/* Definicje graczy */
set P1S := a b c;
set P2S := 1 2 3 4;

/* Macierz wypłat gracza P1 */
param Payoff1
    :  1  2  3  4 :=
    a  0  3  1  8
    b  5  5  4  6
    c  2  6  0  5;

/* Macierz wypłat gracza P2 */
param Payoff2
    :  1  2  3  4 :=
    a  6  6  2  2
    b  3  2  5  6
    c  0  7  0  1;
 
end;
\end{mathprog}

\begin{mathprog}[caption=Szukanie rozwiązania]
set P1S;
set P2S;

param Payoff1{P1S, P2S};
param Payoff2{P1S, P2S};

param U1 := max{i in P1S, j in P2S}
  Payoff1[i,j] - min{i in P1S, j in P2S} Payoff1[i,j];
param U2 := max{i in P1S, j in P2S}
  Payoff2[i,j] - min{i in P1S, j in P2S} Payoff2[i,j];

var b1s{P1S}, binary;
var b2s{P2S}, binary;

var p1s{P1S}, >= 0, <=1;
var p2s{P2S}, >= 0, <=1;

var u1s{P1S};
var u2s{P2S};

var r1s{P1S};
var r2s{P2S};

var u1;
var u2;

/* Przykładowa funkcja celu */
minimize AvarageRegret:
  ( sum{i in P1S} r1s[i] + sum{j in P2S} r2s[j] ) / 2;

/* Ograniczenie (1) z definicji problemu */
s.t. Probabilities1:
	sum{i in P1S} p1s[i] = 1;
s.t. Probabilities2:
	sum{j in P2S} p2s[j] = 1;

/* Ograniczenie (2) z definicji problemu */
s.t. UtilitiesValue1{i in P1S}:
	u1s[i] = sum{j in P2S} p2s[j] * Payoff1[i,j];
s.t. UtilitiesValue2{j in P2S}:
	u2s[j] = sum{i in P1S} p1s[i] * Payoff2[i,j];

/* Ograniczenie (3) z definicji problemu */
s.t. MaxUtilities1{i in P1S}:
	u1 >= u1s[i];
s.t. MaxUtilities2{j in P2S}:
	u2 >= u2s[j];

/* Ograniczenie (4) z definicji problemu */
s.t. Regret1{i in P1S}:
	r1s[i] = u1 - u1s[i];
s.t. Regret2{j in P2S}:
	r2s[j] = u2 - u2s[j];

/* Ograniczenie (5) z definicji problemu */
s.t. Probability1{i in P1S}:
	p1s[i] <= 1 - b1s[i];
s.t. Probability2{j in P2S}:
	p2s[j] <= 1 - b2s[j];

/* Ograniczenie (6) z definicji problemu */
s.t. RegretAndMax1{i in P1S}:
	r1s[i] <= U1 * b1s[i];
s.t. RegretAndMax2{j in P2S}:
	r2s[j] <= U2 * b2s[j];

solve;
 
/* Wyświetlenie wyników */ 

printf "Player 1 strategies:\n";
for{i in P1S}
	printf "%s -> %s\n", i, p1s[i];
printf "Sum = %s, \n", sum{i in P1S} p1s[i] * u1s[i];
printf "Max = %s, \n", U1;
 
printf "Player 2 strategies:\n";
for{j in P2S}
	printf "%s -> %s\n", j, p2s[j];
printf "Sum = %s, \n", sum{j in P2S} p2s[j] * u2s[j];
printf "Max = %s, \n", U2;
\end{mathprog}

Dla testów wybraliśmy następującą funkcję celu:
$$\textbf{minimalizuj} \frac{\sum_{s_1 \in S_1} r_{s_1} + \sum_{s_2 \in S_2} r_{s_2}}{2}$$
minimalizującą średnią wartość żalu przy założeniu rozkładu jednostajnego. Możliwym usprawnieniem jest zmiana na któreś
z rozwiązań zaproponowanych przez autorów oryginalnej pracy.

\subsection{Gry w formie ekstensywnej}

Rozważamy wyłącznie gry z pełną informacją. Aby zaleźć w nich równowagę Nasha rekurencyjnie obliczamy wartości
podgier dla każdego z wierzchołków. W wywołaniach przekazujemy listę na którą dopisujemy optymalny wybór dla danego
gracza. Po obliczeniu wartości dla wszystkich strategii, wybieramy najlepszą, dopisujemy ją do listy i zwracamy wartość
gry. Po zakończeniu otrzymujemy listę wybranych strategii (równowagę Nasha) oraz wypłaty graczy.

\section{Parser danych wejściowych}

Za przetwarzanie danych wejściowych odpowiada parser napisany przy pomocy pakietów \LIB{Flex} oraz \LIB{Bison}. Ponieważ
przy ich udziale tworzy się parsery typu LALR, możliwe jest pisane przejrzystej, łatwej w rozwijaniu gramatyki, zaś 
wyjściowy kod jest wydajny i bezpieczny. Przyjętą gramatykę przedstawiono na listingu \ref{GRAMMAR}.

\begin{code}[caption=Gramatyka,
             label=GRAMMAR,
             literate={->}{$\rightarrow$}{2}
                      {empty}{$\epsilon$}{1}]
program          -> statements EOF
statements       -> statements statement | empty
statement        -> context_execute; | context_load; |
                    context_save; | definition; | query;
context_execute  -> "EXECUTE" identifier
context_load     -> "LOAD" identifier
context_save     -> "SAVE" "TO" identifier
definition       -> "LET" identifier "BE" object
query            -> "FIND" identifiers
                     "FOR" objects opt_conditions
objects          -> objects, object | object
object           -> game | player | param
game             -> "STRATEGIC" "GAME" details "END" |
                    "EXTENSIVE" "GAME" details "END"
details          -> "WITH" objects "SUCH" "AS" data
player           -> PLAYER identifier { identifiers }
param            -> identifier | number
params           -> params, param | param
identifiers      -> identifiers, identifier |
                    identifier
opt_conditions   -> condition_collection | empty
conditions       -> conditions, condition |
                    "WITH" condition
condition        -> "PLAYER" object "CHOOSE" object |
                    "PLAYER" object "IN" { objects } |
                    "PLAYER" object "AT" object
                                    "CHOOSE" object |
                    "PLAYER" object "AT" object
                                    "IN" { objects }
data             -> data_coordinates
data_coordinates -> data_coordinates, data_coordinate |
                    data_coordinate
data_coordinate  -> { coordinates: data_coordinates } |
                    { coordinates: params }
coordinates      -> coordinates, coordinate |
                    coordinate
coordinate       -> identifier = identifier
\end{code}

W tak zdefiniowanej gramatyce \textit{coordinate} reprezentuje wybór gracza (w strategiach czystych dla gier w postaci
normalnej, w danym ruchu dla gier w postaci ekstensywnej), \textit{coordinates} odpowiadają profilowi strategii dla
kilku graczy, zaś \textit{data} i \textit{data\_coordinates} wiążą profile/sekwencje ruchów z odpowiadającymi im
wypłatami.

\textit{player} definiuje gracza - zapisuje jego nazwę do której można się odnosić przy definiowaniu profilów, z listą
dostępnych dla niego strategii.

\textit{game} oraz \textit{details} definiują grę łącząc definicje graczy, wraz z wartościami wypłat dla poszczególnych
profili/sekwencji ruchów (\textit{data}).

\textit{param} odpowiada liczbom (\textit{number}) i identyfikatorom zmiennych (\textit{identifier}) podczas wykonywania
zapytań, \textit{object} to reprezentacja każdego obiektu na którym możemy wykonać zapytania - obiektami tymi są gry,
gracze oraz parametry.

Definicje te są z kolei używane przy definiowaniu zmiennych (\textit{definition}) oraz zapytaniach (\textit{query}).

\textit{program} składa się z sekwencji instrukcji (\textit{statement}) zakończonych średnikiem.

Gramatyka przewiduje również dodatkowe możliwości języka, takie, jak definiowanie warunków przy zapytaniach o równowagi,
jak również wczytywanie i zapisywanie danych do plików. Budowanie parsera LALR rozpoznającego tą gramatykę nie prowadzi
do konfliktów redukcji-przesunięcia. Przykłady rozpoznawanej składni można znaleźć w Dodatku \ref{GTL}.

\section{Praktyki przyjęte w projekcie}

W projekcie przyjęto pewne zasady pozwalające na lepszą organizację oraz łatwiejsze poruszanie się po kodzie. Przede
wszystkim projekt podzielony został na moduły:
\begin{itemize}
\item \MODULE{model} - zawiera definicje modeli, oraz metod dostępowych do przechowywanych w nich danych,
\item \MODULE{routines} - zawiera definicje algorytmów obliczających równowagi w poszczególnych rodzajach gier,
\item \MODULE{gtl} - zawiera definicje parsera oraz klas reprezentujących poszczególne wierzchołki drzewa składniowego,
\item \MODULE{program} - zawiera definicje oraz funkcje używane przy uruchamianiu programu.
\end{itemize}

Projekt został również podzielony na 2 nadrzędne katalogi:
\begin{itemize}
\item \MODULE{include} - zawiera publiczne nagłówki, które powinny być widoczne dla programistów chcących korzystać
wyłącznie z biblioteki ale nie z programu wykonywalnego. Większość deklaracji stanowią interfejsy oraz fabryki
pozwalające uzyskać dostęp do implementacji, zadeklarowanych klas,
\item \MODULE{src} - zawiera wewnętrzne nagłówki oraz implementacje publicznych interfejsów.
\end{itemize}
Ponieważ większość klas jest udostępniana wyłączne jako interfejsy, użytkownicy nie są zależni od większości zmian
w wewnętrznym uporządkowaniu programu, wykorzystanych algorytmach oraz innych szczegółach implementacji podatnych
na zmiany.

\end{document}
