\documentclass[polish]{standalone}
\usepackage{thesis}

\begin{document}
\pagestyle{headings}

\chapter{Wnioski}

W trakcie rozwijania projektu udało mi się zaimplementować procedury obliczające równowagi Nasha w kilku przypadkach
gier. Procedura obliczająca równowagi w grach macierzowych to przykład zwykłego programowania liniowego. Do jej
rozwiązania wystarczył algorytm simpleksowy. W grach dwumacierzowych wykorzystałem programowanie całkowitoliczbowe,
przez co ponownie mogłem wykorzystać wspomniany algorytm. Równowagi czyste odnajdywałem przeszukując przestrzenie
rozwiązań: iteracyjnie - w grach w postaci strategicznej - oraz rekurencyjnie - w grach w formie ekstensywnej z pełną
informacją.

Nie udało mi się niestety zaimplementować algorytmów szukających równowag w wieloosobowych grach w postaci normalnej.
Opisałem zaledwie jeden z algorytmów, które mogą do tego celu posłużyć. Nie rozważyłem również algorytmów szukania
równowag w grach w postaci ekstensywnej w strategiach innych niż czyste oraz bez pełnej informacji. Projekt zrealizował
jedynie podstawową wymaganą funkcjonalność.

Przedstawiając znane obecnie algorytmy zaważyłem pewne drobne podobieństwa w rozumowaniu autorów. W grach jedno- i
dwyumacierzowych (Lemke-Howson) aby znaleźć równowagę przeszukujemy wierzchołki pewnego wielowymiarowego wypukłego
wielościanu (sympleksu), wiedząc, że równowaga będzie się znajdować w jednym z wierzchołków. O ile jednak w grach
macierzowych o wyborze następnego sprawdzanego wierzchołka decyduje funkcja celu, w grach dwumacierzowych Lemke oraz
Howson szukają parametrów spełniających problem komplementarności liniowej. Podobne przeszukiwanie pojawia się w
algorytmie Portera-Nudelmana-Shohama - w nich jednak o kolejności sprawdzania rozwiązań decyduje nie incydentność
wierzchołków sympleksa ale heurystyka preferująca profile o mniejszym i bardziej zbalansowanym wsparciu. Możemy więc
przypuszczać, że autorzy tych algorytmów opierali się oni na podobnych koncepcjach, ale wyraźne różnice pokazują, że są
to mimo wszystko odrębnie rozwijane rozwiązania.

Nie znalazłem podobieństw miedzy ww. a algorytmem Sandholma-Gilpina-Contizera. Nie jestem też wstanie wypowiedzieć się
na temat podobieństw algorytmów stosowanych w grach w postaci ekstensywnej.

Dalszym kierunkiem rozwoju prac nad projektem mogłoby być wdrożenie algorytmów obliczających równowagi strategiach
mieszanych w wieloosobowych grach w postaci normalnej. Użytą funkcję celu w algorytmie Sandholma-Gilpina-Contizera można
by zastąpić jednym z proponowanym przez nich rozszerzeń. Podobnie można by wprowadzić wsparcie dla gier w postaci
ekstensywnej bez pełnej informacji i obliczać dla nich równowagi w strategiach mieszanych oraz postępowania.

Tak więc, z powodzeniem udało mi się stworzyć narzędzie obliczające równowagi w podstawowych wymaganych rodzajach gier.
W rozważanych przypadkach rozwiązania opierają się na algorytmie simpleks implementowanym przez bibliotekę GLPK lub
przeszukiwaniu przestrzeni rozwiązań. Możliwy jest dalszy rozwój projektu w celu wspierania większej ilość rodzajów
gier.

\end{document}