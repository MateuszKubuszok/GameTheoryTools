\documentclass{beamer}

\usepackage[utf8]{inputenc} % input encoding
\usepackage[T1]{fontenc}    % 8-bit font encoding (needed by e.g. PDFs)
\usepackage{polski}         % i18n package
\usepackage[polish]{babel}  % i18n package
\usepackage{makeidx}        % indexing package

\usepackage{listings}       % actual code package

\usetheme{Median}

\lstset{
  language={[Sharp]C},
  basicstyle=\tiny
}

\title[Narzędzia do równowag Nasha]{Narzędzia do obliczania równowag Nasha}
\author[M. Kubuszok]{Mateusz Kubuszok}
\institute[Politechnika Wrocławska]{
  Promotor\\
  Dr inż. Piotr Więcek\\[1ex]
}
\date[Marzec 2014]{4.3.2014}

\begin{document}

\begin{frame}[plain]
  \titlepage
\end{frame}

\begin{frame}{Cel}
Narzędzia do obliczania pewnych równowag Nasha:
\begin{itemize}
\item otwarte oprogramowanie,
\item projekt zorientowany obiektowo,
\item szybki w działaniu,
\item łatwy w utrzymaniu i rozbudowie,
\item dobrze udokumentowany.
\end{itemize}
\end{frame}

\begin{frame}{Istniejące rozwiązania}
\begin{itemize}

\item pakiet Gambit
\begin{itemize}
\item duże możliwości,
\item mało przyjazny interfejs CLI,
\item słabo udokumentowany.
\end{itemize}

\end{itemize}
\end{frame}

\begin{frame}{Implementacja}
Zaimplementowano algorytmy szukania równowag:
\begin{itemize}
\item dla strategii czystych w grach strategicznych,
\item dla strategii mieszanych w jednomacierzowych (LP),
\item dla strategii mieszanych w dwumacierzowych (MIP),
\item dla strategii czystych w grach ekstensywnych.
\end{itemize}
\end{frame}

\begin{frame}{Narzędzia}
Wykorzystane narzędzia:
\begin{itemize}
\item język C++11,
\item biblioteka Boost,
\item biblioteka GLPK,
\item programy Flex oraz Bison.
\end{itemize}
\end{frame}

\begin{frame}[fragile]{Przykład: gra dwumacierzowa}
\begin{lstlisting}
LET player1 BE
  PLAYER p1 { s1, s2 };

LET game1 BE
  STRATEGIC GAME
  WITH
    player1,
    PLAYER p2 { s1, s2 }
  SUCH AS
    { p1=s1, p2=s1 : 10, 20 },
    { p1=s1, p2=s2 : 30, 40 },
    { p1=s2 :
      { p2=s1 : 50, 60 },
      { p2=s2 : 70, 80 }
    }
  END;

FIND pure_equilibrium, mixed_equilibrium FOR game1;
\end{lstlisting}
\end{frame}

\begin{frame}[fragile]{Przykład: gra dwumacierzowa}
\begin{lstlisting}
1:
    pure_equilibrium:
        Pure Strategies:
            p1:
                s2
            p2:
                s2
        Payoff:
                    p1, p2
            Payoff:
                    70, 80
    mixed_equilibrium:
        Mixed Strategies:
            p1:
                s1:
                    0.333333
                s2:
                    0.666667
            p2:
                s1:
                    0.666667
                s2:
                    0.333333
        Payoff:
                    p1, p2
            Payoff:
                    43.3333, 53.3333
\end{lstlisting}
\end{frame}

\begin{frame}[fragile]{Przykład: gra w postaci ekstensywnej}
\begin{lstlisting}
LET game2 BE
  EXTENSIVE GAME
  WITH
    player1,
    PLAYER p2 { s1, s2 }
  SUCH AS
    { p1=s1 :
      { p2=s1 : 10, 20 },
      { p2=s2 : 30, 40 }
    },
    { p1=s2 :
      { p2=s1 : 50, 60 },
      { p2=s2 : 70, 80 }
    }
  END;

FIND pure_equilibrium FOR game2;
\end{lstlisting}
\end{frame}

\begin{frame}[fragile]{Przykład: gra w postaci ekstensywnej}
\begin{lstlisting}
1:
    pure_equilibrium:
        Pure Strategies:
            p1:
                1:
                    s2
            p2:
                1:
                    s2
                2:
                    s2
        Payoff:
                    p1, p2
            Payoff:
                    70, 80
\end{lstlisting}
\end{frame}

\end{document}