\documentclass{standalone}
\usepackage{thesis}

\begin{document}
\appendix

\chapter{Wybrane fragmenty programu}

TODO

\chapter{Kompilacja programów}

\section{Wymagania}

Program został napisany z wykorzystaniem:
\begin{itemize}
\item kompilatora \PROG{g++} w wersji \textit{4.8} do kompilacji źródeł, możliwego do zastąpienia przez
\PROG{Clang++} \textit{3.3},
\item pakietu \LIB{Bison} w wersji 2.5,
\item pakietu \LIB{flex},
\item biblioteki \LIB{GLPK},
\item biblioteki \LIB{Boost} w wersji co najmniej \textit{1.48} (\LIB{g++}) lub \textit{1.50.0-beta1}
(\LIB{Clang++}),
\item biblioteki \LIB{GMP},
\item środowiska do budowania projektów \PROG{Scons}
\end{itemize}
konieczne jest więc zainstalowanie ich przed podjęciem próby zbudowania projektu.

\subsection{Unix i uniksopodobne}

W środowiskach uniksowych służą do tego zazwyczaj programy do zarządzania pakietami, które umożliwiają prostą
i bezpieczną instalację programów wraz z zależnościami.

\subsection{Windows}

Na systemie Windows program był testowany w środowisku \PROG{Cygwin}, trzeba jednak zaznaczyć, że nie posiadało
ono pełnej wersji biblioteki \LIB{Boost} i konieczne było ręcznie zbudowanie jej w wymaganej wersji/ręczne
instalowanie plików binarnych.

\section{Budowanie}

Mając zainstalowane wszystkie wymagane biblioteki należy uruchomić linię komend oraz przejść do katalogu w którym
przechowujemy kod źródłowy projektu. Następnie należy uruchomić polecenie \CODE{scons} aby zbudować wszystkie
pliki binarne (dostępne w katalogu \CODE{\\bin}):
\begin{itemize}
\item \PROG{gtl\_program} - plik wykonywalny parsera danych,
\item \PROG{ModelTests} - testy jednostkowe modułu \textbf{model},
\item \PROG{RoutinesTests} - testy jednostkowe modułu \textbf{routines},
\item \PROG{GTLTests} - testy jednostkowe modułu \textbf{gtl},
\item \PROG{ProgramsTests} - testy jednostkowe modułu \textbf{program},
\item \LIB{libGTT.a} - biblioteka statyczna projektu,
\item \LIB{libGTT.so} - biblioteka dynamiczna projektu.
\end{itemize}

\chapter{Instrukcja obsługi biblioteki}

TODO

\chapter{Instrukcja obsługi parsera}

TODO

\end{document}
