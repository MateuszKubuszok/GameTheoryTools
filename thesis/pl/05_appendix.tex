\documentclass[polish]{standalone}
\usepackage{thesis}

\begin{document}
\appendix

\chapter{Kompilacja programów}

\section{Wymagania}

Program został napisany z wykorzystaniem:
\begin{itemize}
\item kompilatora \PROG{g++} w wersji \VER{4.8} do budowania ze źródeł, możliwego do zastąpienia przez \PROG{clang++} 
\VER{3.3},
\item pakietu \LIB{Bison} w wersji \VER{2.5} (na platformie \PROG{Cygwinie} wymaga również biblioteki \LIB{M4}),
\item pakietu \LIB{flex},
\item biblioteki \LIB{GLPK},
\item biblioteki \LIB{Boost} w wersji co najmniej \VER{1.48} (\PROG{g++}) lub \VER{1.50.0-beta1} (\PROG{clang++}),
\item biblioteki \LIB{GMP},
\item środowiska do budowania projektów \PROG{Scons},
\end{itemize}
konieczne jest więc zainstalowanie ich przed podjęciem próby zbudowania projektu.

\subsection{Unix i uniksopodobne}

W środowiskach uniksowych służą do tego zazwyczaj programy do zarządzania pakietami, które umożliwiają prostą
i bezpieczną instalację programów wraz z zależnościami.

\subsection{Windows}

Na systemie Windows program był testowany w środowisku \PROG{Cygwin}, trzeba jednak zaznaczyć, że nie posiadało ono
w repozytorium pełnej biblioteki \LIB{Boost} i konieczne było ręczne zbudowanie jej w wymaganej wersji lub instalowanie
jej plików binarnych.

\section{Budowanie}

Mając zainstalowane wszystkie wymagane biblioteki, należy uruchomić linię komend oraz przejść do katalogu, w którym
przechowujemy kod źródłowy projektu. Następnie należy uruchomić polecenie \CODE{scons} aby zbudować wszystkie pliki
binarne (dostępne w katalogu \CODE{bin/release} lub \CODE{bin/debug}):
\begin{itemize}
\item \PROG{gtl\_program} - plik wykonywalny parsera danych,
\item \PROG{ModelTests} - testy jednostkowe modułu \textbf{model},
\item \PROG{RoutinesTests} - testy jednostkowe modułu \textbf{routines},
\item \PROG{GTLTests} - testy jednostkowe modułu \textbf{gtl},
\item \PROG{ProgramsTests} - testy jednostkowe modułu \textbf{program},
\item \LIB{libGTT.a} - biblioteka statyczna projektu,
\item \LIB{libGTT.so} - biblioteka dynamiczna projektu.
\end{itemize}
Dokumentacja zawiera dodatkowe wskazówki i sugestie, pozwalające wybierać budowane pliki oraz przyspieszać proces 
budowania np. poprzez urównoleglenie kompilacji.

\chapter{Instrukcja obsługi biblioteki}

Po zbudowaniu biblioteka jest typową biblioteką języka \LANG{C++}. Oznacza to, że można ją dowolnie linkować do każdego
programu pisanego w języku \LANG{C++}, tak długo jak zapewniona będzie binarna kompatybilność między kompilatorem użytym
do zbudowania biblioteki, a tym wykorzystanym do kompilacji programu. W przypadku \PROG{g++} zgodność taka zachodzi
między większością wersji (złamano ją tylko w 2 przypadkach w ciągu ostatnich 10 lat), zaś kompilator \PROG{Clang} stara
się zachować zgodność z \PROG{GCC} na poziomie zarówno binarnym jak i API.

W obu wyżej wymienionych przypadkach wystarczy podczas linkowania dodać parametr \CODE{-lścieżka\_do\_biblioteki},
aby linker mógł dołączyć bibliotekę do programu wykonywalnego.

Aby korzystać ze wszystkich obiektów i metod biblioteki, konieczne jest również dołączenie nagłówków definiujących
dane obiekty. Należy je dołączyć do ścieżki przeszukiwania poprzez parametr \CODE{-iścieżka\_do\_katalogu\_include}
wskazujący na katalog \CODE{include}. Wówczas wystarczy użyć makra \CODE{\#incude} wskazującego na jeden z poniższych
plików:
\begin{itemize}
\item \CODE{"gt.hpp"} - zawiera wyłącznie definicje wspólne dla wszystkich modułów,
\item \CODE{"gt/version.hpp"} - zawiera wyłącznie numer wersji biblioteki,
\item \CODE{"gt/model.hpp"} - definicje interfejsów wykorzystywanych przez moduł \MODULE{modelu}, a także fabryk
pozwalających uzyskać do stęp do ich implementacji,
\item \CODE{"gt/routines.hpp"} - definicje interfejsów wykorzystanych przez moduł \MODULE{routines}, fabryk 
udostępniających ich implementacje. Odnosi się również do definicji wykorzystanych w module \MODULE{model},
\item \CODE{"gt/gtl.hpp"} - definiuje interfejsy oraz fabryki modułu \MODULE{GTL}, oraz odniesienia do dwóch poprzednich
modułów,
\item \CODE{"gt/program.hpp"} - deklaracje modułu \MODULE{program} oraz pozostałych modułów.
\end{itemize}

Przykładowo aby używać definicji modułu \MODULE{GTL} należy użyć dyrektywy \CODE{\#include "gt/gtl.hpp"}.

Szczegóły wykorzystania poszczególnych klas dostępne są w dokumentacji wygenerowanej przy pomocy programu
\PROG{Doxygen}. Przygotowano jej 3 konfiguracje do uruchomienia z katalogu głównego projektu, umieszczone w podkatalogu
\CODE{doxygen}. Uruchamiając poszczególne konfiguracje uzyskamy:
\begin{itemize}
\item \CODE{doxygen doxygen/tutorial\_} - instrukcje kompilacji i uruchamiania projektu,
\item \CODE{doxygen doxygen/public\_api} - dokumentacja publicznego API projektu, z której należy korzystać przy 
normalnym użytkowaniu biblioteki,
\item \CODE{doxygen doxygen/private\_api} - pełna dokumentacja projektu, zawierająca opisy wewnętrznych klas i 
interfejsów projektu. Użyteczna głównie w celu rozwijania i ulepszania biblioteki.
\end{itemize}

Użyteczne w zrozumieniu działania poszczególnych metod mogą być ich testy jednostkowe - dostarczają one minimalny wgląd
w przypadki użycia poszczególnych klas oraz ich metod, a na podstawie ich kodu użytkownik biblioteki powinien być w
stanie zrozumieć jak użyć ich własnoręcznie w swoim programie.

\chapter{Instrukcja obsługi parsera}

\section{Uruchamianie}

Parser zdefiniowany jest w module \CODE{GTL}, a sterowany za pośrednictwem modułu \CODE{program}. Aby z niego skorzystać
powinniśmy uruchomić plik \PROG{gtl\_program}. Jako domyślne strumienie przyjmuje on standardowe wejście, wyjście
oraz wyjście błędów. Możliwe jest jednak wczytanie z/pisanie do pliku:
\begin{itemize}
\item \CODE{gtl\_program -i plik} - wczytuje wejście z podanego pliku,
\item \CODE{gtl\_program -o plik} - wypisuje wyjście do podanego pliku,
\item \CODE{gtl\_program -e plik} - wypisuje wyjście błędów do podanego pliku.
\end{itemize}

Parser potrafi wyświetlać dane w jednym z 3 formatów: zwykły tekst (format domyślny), \CODE{XML} oraz \CODE{JSON}. Aby
zmienić tryb wyświetlania należy uruchomić program z odpowiednim parametrem:
\begin{itemize}
\item \CODE{gtl\_program -r PLAIN} - zwykły tekst (domyślny format),
\item \CODE{gtl\_program -r XML} - format \CODE{XML},
\item \CODE{gtl\_program -r JSON} - format \CODE{JSON}.
\end{itemize}

Dodatkowo program można uruchomić z parametrami:
\begin{itemize}
\item \CODE{gtl\_program --version} - wyświetla numer wersji,
\item \CODE{gtl\_program --help} - wyświetla pomoc odnośnie uruchomienia programu.
\end{itemize}

\section{Format danych wejściowych \label{GTL}}

Język \LIB{GTL} operuje na prostych obiektach, które można zapytać o jakąś ich własność. Można się do nich odnosić przez
wartość jak również przez zdefiniowaną nazwę. Po wpisaniu polecenia:

\begin{code}
LET name BE value;
\end{code}

Pod nazwą \CODE{name} dostępna będzie wartość \CODE{value}. Zapytania wyglądają następująco:

\begin{gtl}
FIND property[, property2[, ...]]
 FOR value[, value2[, ...]];
\end{gtl}

dzięki czemu uzyskamy właściwości \CODE{property}, \CODE{propert2}, ... dla obiektów \CODE{value}, \CODE{value2}, ...

Zdefiniowane imiennie dane dostępne są w ramach pewnego kontekstu (zasięgu). Gdy uruchamiany program, inicjalizujemy 
najwyższy kontekst i wszystkie zmienne jakie zdefiniujemy w konsoli, będą do niego trafiać. Może on jednak posiadać
konteksty potomne - uruchamiając polecenie:

\begin{gtl}
EXECUTE "plik.gtl";
\end{gtl}

tworzymy nowy kontekst dla pliku \CODE{file}, do niego trafią wszystkie definiowane zmienne i gdy wczytywanie pliku
dobiegnie końca, zostanie on zniszczony wraz ze wszystkimi zmiennymi jakie w nim zdefiniowano. Posiadać on będzie dostęp
do wszystkich zmiennych zdefiniowanych w kontekście z którego uruchomiono plik, a w przypadku konfliktu nazw, zmienna
oryginalna zostanie przesłonięta jedynie w kontekście potomnym. Dla odmiany polecenie:

\begin{gtl}
LOAD "plik.gtl";
\end{gtl}

wczyta i wykona plik w bieżącym kontekście.

Ostatnim poleceniem jest:

\begin{gtl}
STORE "plik.gtl";
\end{gtl}

zapisujące wszystkie widziane w danym kontekście zmienne do pliku.

Na chwilę obecną format GTL pozwala wyłącznie na samodzielne wykonywanie tych 4 poleceń - tworzenie obiektów może 
następować wyłącznie w ramach polecenie \CODE{LET ... BE ...;} lub \CODE{FIND ... FOR ...;}.

\section{Gracze}

Składnia tworzenia obiektu gracza wygląda następująco:

\begin{gtl}
PLAYER name { strategy1[, strategy2[, ...]] }
\end{gtl}

definiuje ona gracza \CODE{name}, o strategiach czystych \CODE{strategy1}, \CODE{strategy2}, ... W praktyce, występuje
ona głównie w połączeniu z poleceniem \CODE{LET ... BE ...;}:

\begin{gtl}
LET player1 BE PLAYER { strategy1, strategy2 };
\end{gtl}

lub z \CODE{FIND ... FOR ...;}:

\begin{gtl}
FIND name, strategies
 FOR PLAYER name1 { strategy1, strategy2 };
\end{gtl}

\section{Gry w postaci strategicznej}

Każda gra w postaci strategicznej powinna mieć zdefiniowane wartości funkcji wypłaty dla wszystkich profili strategii
czystych poszczególnych graczy. Pozwala jednak na pewną dowolność w definiowaniu ich:

\begin{gtl}
LET game1 BE
  STRATEGIC GAME
  WITH
    player1, // gracz zdefiniowany wcześniej oraz
    PLAYER player2 { strategy1, strategy2 } // ad hoc
  SUCH AS
    { player1=strategy1: // częściowa definicja
     { player2=strategy1: 10, 20 }, // uzupełnienie
     { player2=strategy2: 30, 40 }
    },
    // pełna definicja
    { player1=strategy2, player2=strategy1: 50, 60 },
    { player1=strategy2, player2=strategy2: 70, 80 }
  END;
\end{gtl}

Możliwe jest więc zarówno przekazywanie definicji graczy wg nazwy jak i podawanie ich jawnie. Profile strategii z
kolei mogą być podawane od razu w komplecie, jak również rozbite na etapy. Każda definicja strategii ma postać
\CODE{\{ gracz=strategia[, gracz2=strategia2[, ...]] : [kolejne definicje lub wypłaty graczy] \}}. Wymagane jest podanie
wszystkich wypłat, natomiast niedozwolone dwukrotne definiowanie wypłaty dla tego samego profilu strategii.

Dla każdej gry można podjąć próbę obliczenia równowagi w strategiach czystych

\begin{gtl}
FIND pure_equilibrium FOR game1;
\end{gtl}

Dla gier jednomacierzowych (tutaj: gier 2-osobowych dla których każdy profil ma zdefiniowane 2 przeciwne wypłaty albo
wręcz tylko jedną) oraz dwumacierzowych (gier 2-osobowych gdzie każdy profil ma zdefiniowane 2 wypłaty) można dodatkowo
próbować obliczyć równowagę w strategiach mieszanych:

\begin{gtl}
FIND mixed_equilibrium FOR game1;
\end{gtl}

\section{Gry w postaci ekstensywnej}

Gry w postaci ekstensywnej definiuje się niemal identycznie jak gry w postaci strategicznej. Jedyną różnicą jest
ograniczenie, że struktura definicji musi odpowiadać strukturze drzewa gry:

\begin{gtl}
LET game2 BE
  EXTENSIVE GAME
  WITH
    player1, // gracz zdefiniowany wcześniej oraz
    PLAYER p2 { s1, s2 } // zdefiniowany ad hoc
  SUCH AS      // struktura odpowiada drzewu:
    { p1=s1 :  // w każdym wierzchołku, decyzję podejmuje
      { p2=s1 : 10, 20 }, // jeden gracz,
      { p2=s2 : 30, 40 }  // liście zawierają wypłaty
    },
    { p1=s2 :
      { p2=s1 : 50, 60 },
      { p2=s2 : 70, 80 }
    }
  END;
\end{gtl}

Dla tak zdefiniowanego drzewa możemy poszukać równowagi w strategiach czystych:

\begin{gtl}
FIND pure_equilibrium FOR game2;
\end{gtl}

\end{document}
