\documentclass{standalone}
\usepackage{thesis}

\begin{document}
\pagestyle{headings}

\chapter{Teoria}

\section{Wprowadzenie}

Przez \textit{grę} rozumiemy matematyczny opis pewnej sytuacji konfliktowej, w której biorą udział określone strony,
nazywane \textit{graczami}. Każdej stronie przypisujemy pewien zbiór \textit{strategii}, z których w danym momencie
wykorzystuje jedną. Każdemu z graczy przypisujemy również \textit{funkcję wypłaty}, której wartość uzależniona jest
od wszystkich strategii granych przez graczy w określonej chwili.

W zależności od rozpatrywanego rodzaju gry formalna definicja wymienionych pojęć jest nieco inna. W tej pracy rozważane
będą 2 rodzaje gier: \textit{gry w postaci strategicznej} oraz \textit{gry w postaci ekstensywnej}. Dla obu z nich możliwe
jest zdefiniowanie tzw. \textit{równowagi Nasha} tj. takiej kombinacji wyborów strategii przez graczy, aby dla każdego z nich
nieopłacalne było odstąpienie od obecnie granej strategii.

Szczególną uwagę poświęcono grom dla których możliwe jest obliczenie równowagi Nasha przy pomocy programowania
liniowego/całkowitoliczbowego lub rekurencyjnego przeszukania danych.

\section{Gra w postaci strategicznej}

\begin{definition}
\textbf{Grą w postaci strategicznej (normalnej)} nazywamy trójkę:
\begin{enumerate}
\item skończony zbiór graczy $I = 1, 2, 3, ..., n$,
\item przestrzeń strategii czystych gracza $S_i$ zdefiniowaną dla każdego gracza $i \in I$,
\item funkcję wypłaty $u_i(s)$ określającą wypłatę gracza $i \in I$ dla każdego profilu $s \in S$.
\end{enumerate}
\end{definition}

Przy tak zdefiniowanej grze zakładamy, że każdy z graczy decyduje się na jedną ze swoich strategii, a następnie
na podstawie wyborów podjętych przez wszystkich graczy przypisujemy każdemu z nich wypłatę.

Możemy przyjąć, że dla tak przyjętych oznaczeń $s_i$ oznacza wybór $i$-tego gracza dla profilu $s$, zaś $s_{-i}$ profil
wszystkich wyborów oprócz profilu gracza $i$. Analogicznie $S_{-i}$ oznacza zbiór wszystkich możliwych odpowiedzi na granie
strategii $s_i$ przez gracza $i$.

\begin{definition}
\textbf{Strategią mieszaną} gracza $i$ nazywamy rozkład $\sigma_i$ nad strategiami czystymi $S_i$. Wówczas funkcja wypłaty
ma postać $u_i(\sigma_i, \sigma^{*})$ i jest wartością oczekiwaną funkcji wypłaty dla strategii mieszanych wszystkich graczy.
\end{definition}

\subsection{Równowagi w strategiach czystych}

\begin{definition}
\textbf{Równowagą Nasha w strategiach czystych} w grze w postaci strategicznej nazywamy profil strategii $s \in S$ taki, że:
$$\forall_{i \in I} \forall_{s_{-i} \in S_{-i}} u_i(s) \geq u_i(s_i, s_{-i})$$
\end{definition}

Oznacza to, że równowagą jest profil strategii, dla którego pojedyncza zmiana granej strategii nie jest w stanie poprawić wypłaty
żadnego z graczy.

Równowaga taka nie zawsze istnieje, czego przykładem jest gra w papier-kamień-nożyce:

\begin{tabular}[t]{| c | c | c | c |}
\hline
$p_1 \downarrow$, $p_2 \rightarrow$  & papier & kamień & nożyce \\
\hline
papier                               & 0,0    & 1,-1   & -1,1   \\
\hline
kamień                               & -1,1   & 0,0    & 1,-1   \\
\hline
nożyce                               & 1,-1   & -1,1   & 0,0    \\
\hline
\end{tabular}

W takiej sytuacji każdy z graczy zawsze może zmienić swoją strategię na taką, w której jego wypłata zostanie zwiększona.

\subsection{Równowagi w strategiach mieszanych w grach jednomacierzowych}

TODO

\subsection{Równowagi w strategiach mieszanych w grach dwumacierzowych}

TODO

\section{Gra w postaci ekstensywnej}

TODO

\subsection{Równowagi w strategiach czystych w grach z informacją doskonałą}

TODO

\end{document}
