\documentclass[polish]{standalone}
\usepackage{thesis}

\begin{document}
\pagestyle{headings}

\section*{Wstęp}

Celem pracy było stworzenie zestawu narzędzi do obliczania równowag Nasha w pewnych rodzajach gier skończonych.
Narzędzia te miałyby umożliwić obliczenie ich po wprowadzeniu danych w jakimś przyjętym formacie, np. przy pomocy 
parsera, a następnie zwróceniu wyników użytkownikowi. Jedynymi obecnie dostępnymi narzędziami służącymi do tego celu
jest pakiet \PROG{Gambit}, posiada on jednak dość trudną ograniczoną dokumentację, a jego kod źródłowy choć otwarty,
nie jest w opinii autora łatwy w dalszym rozwijaniu, przez co użytkownicy są uzależnieni od dobrej woli twórców pod
względem otrzymywania wsparcia dla projektu. Dodatkowo pakiet ten jest rozbity na moduły, przez co każdy problem jest
rozwiązywany przy pomocy osobnego narzędzia - może to stanowić problem dla użytkowników oczekujących jednego programu,
którego sposób użycia określają w trakcie jego uruchomienia.

W takiej sytuacji dobrym pomysłem wydaje się stworzenie alternatywy, nastawionej na możliwość łatwego rozwijania
projektu oraz przyjaznego interfejsu umożliwiającego wykorzystanie go nawet mniej zaawansowanym użytkownikom. Aby
projekt taki miał rację bytu powinien oczywiście implementować pewną minimalną funkcjonalność obrazującą jego
potencjalne możliwości. Stworzenie prototypu pozwoliłoby stwierdzić czy taka filozofia projektu przyniosłaby jakąkolwiek
korzyść użytkownikom - w przypadku odpowiedzi twierdzącej mogłoby to stanowić impuls do zmiany filozofii rozwoju innych
narzędzi, chociażby wspomnianego już pakietu Gambit.

W trakcie pisania pracy stworzony został prototyp demonstrujący opisywaną ideę. Zdecydowano się na język \LANG{C++},
który zapewnia wysoką wydajność przy dużej czytelności podczas programowania zorientowanego obiektowo. W przypadku
algorytmów wykorzystujących Simplex, skorzystano z funkcji biblioteki \LIB{GLPK}. Prototyp umożliwia wczytanie danych
przy pomocy parsera, obliczenie rozwiązania zadanego problemu i wyświetlenie wyniku. Posiada zaimplementowane algorytmu
obliczające równowagi Nasha:
\begin{itemize}
\item w grach w postaci strategicznej dla strategii czystych,
\item w grach jednomacierzowych dla strategii mieszanych,
\item w grach dwumacierzowych dla strategii mieszanych,
\item w grach ekstensywnych z informacją doskonałą.
\end{itemize}

Przy pisaniu pracy wykorzystano książki \textbf{Game Theory} (\textit{Fudenberg, Tirole}) \cite{FT-GT},
\textbf{Teoria Gier} (\textit{Owen}) \cite{O-GT} oraz pracę \textbf{Mixed-Integer Programming Methods for Finding Nash
Equilibria} (\textit{Sandholm, Gilpin, Conitzer}) \cite{SCG-NE} jako źródła wiedzy teoretycznej na temat rozważanych
problemów oraz dokumentację języka \LANG{C++} oraz bibliotek \LIB{Boost}, \LIB{GLPK}, \LIB{Bison} i \LIB{Flex} jako
materiały pomocnicze przy faktycznej implementacji wspomnianych procedur.

W poszczególnych rozdziałach opisano szczegóły na temat podstaw teoretycznych oraz realizacji projektu:
\begin{enumerate}
\item w dziale \textbf{Teoria} opisano podstawy teoretyczne stojące za projektem. Wprowadzają one pojęcia \textit{gier},
\textit{równowag Nasha},
\textit{strategii czystych i mieszanych} oraz postaci \textit{strategicznej} oraz \textit{ekstensywnej} gier. Opisuje
również algorytmy służące do obliczenia równowag w rozpatrywanych przypadkach gier,
\item w dziale \textbf{Implementacja} opisano szczegóły realizacji programu, strukturę wewnętrzną oraz przyjęty model
danych, jak również algorytmy nie związane bezpośrednio z teorią gier ale wymagane do poprawnego działania programu,
\item w dziale \textbf{Wnioski} opisano rezultaty otrzymane podczas wdrażania projektu oraz możliwości dalszego jego
rozwoju,
\item załączniki zawierają informacje na temat jego kompilacji i użytkowania.
\end{enumerate}

W ogólności cele pracy można uznać za zrealizowane. Jej wynikiem jest działający program, który posiada wymaganą
funkcjonalność co dowodzi realizowalności przyjętych założeń. Ograniczenia czasowe sprawiły jednak, że nie wszystkie
procedury udało się zaimplementować. Szczegóły zostały uważniej omówione w poszczególnych działach.

\end{document}
