\documentclass[polish]{standalone}
\usepackage{thesis}

\begin{document}
\pagestyle{headings}

\section*{Wstęp}

Celem mojej pracy było stworzenie zestawu narzędzi do obliczania równowag Nasha w pewnych rodzajach gier skończonych
w rozumieniu teorii gier. Narzędzia te miały pobierać definicję gry w zdefiniowanym na potrzeby projektu formacie,
odnajdywać równowagę, przy pomocy jednego z omówionych w tej pracy algorytmów, a następnie zwrócić wyniki użytkownikowi.
W pracy przyjrzałem się istniejącym algorytmom i zaobserwowałem różne sposoby na szukanie równowag Nasha. Różni badacze
próbowali odmiennych podjeść - nie są to kolejne ulepszenia jednego rozwiązania.

Jedynymi obecnie dostępnymi narzędziami spełniającymi postawiony cel są składowe pakietu \PROG{Gambit}, pakiet ten
posiada jednak dość ograniczoną dokumentację, a jego kod źródłowy, choć otwarty, nie wydaje mi się łatwy w rozwijaniu.
Każdy algorytm jest też oddelegowany do osobnego modułu - nie jest możliwe automatyczne rozpoznanie problemu oraz wybranie metody jego rozwiązania.

W takiej sytuacji dobrym pomysłem wydało mi się stworzenie alternatywy, nastawionej na możliwość łatwego rozwijania
projektu oraz przyjaznego interfejsu umożliwiającego wykorzystanie programu nawet mniej zaawansowanym użytkownikom. Aby
projekt taki miał rację bytu, powinien implementować minimalną funkcjonalność obrazującą jego potencjalne możliwości
- za taką przyjęto rozwiązywanie gier 2-osobowych. Stworzenie prototypu pozwoliłoby mi stwierdzić, czy taka filozofia
projektu przyniosłaby jakąkolwiek korzyść.

W trakcie pisania pracy stworzyłem prototyp demonstrujący opisywaną ideę. Zdecydowałem się na użycie języka \LANG{C++},
który umożliwia programowanie z wysoką wydajnością, przy zachowaniu czytelności kodu. W przypadku algorytmów
wykorzystujących programowanie liniowe, skorzystano z funkcji biblioteki \LIB{GLPK}. Prototyp umożliwia wczytanie
danych przy pomocy parsera, obliczenie rozwiązania zadanego problemu i wyświetlenie wyniku. Posiada zaimplementowane
algorytmy obliczające równowagi Nasha w następujących grach:
\begin{itemize}
\item w postaci strategicznej (równowagi w strategiach czystych),
\item jednomacierzowych (równowagi w strategiach mieszanych),
\item dwumacierzowych (równowagi w strategiach mieszanych),
\item w postaci ekstensywnej z informacją doskonałą (równowagi w strategiach czystych).
\end{itemize}
Ustępuje on możliwościami wspomnianemu pakietowi \PROG{Gambit}, między innymi poprzez brak wsparcia dla szukania
równowag:
\begin{itemize}
\item w strategiach mieszanych dla gier w postaci strategicznej dla $n$-graczy,
\item w strategiach czystych dla dowolnych gier w postaci ekstensywnej,
\item w strategiach mieszanych i postępowania w grach w postaci ekstensywnej.
\end{itemize}

W kolejnych rozdziałach opisałem szczegóły na temat podstaw teoretycznych oraz realizacji projektu:
\begin{enumerate}
\item w dziale \textbf{Teoria} opisałem podstawy teoretyczne stojące za projektem. Znajdują się tam definicje
\textit{gier}, \textit{równowag Nasha}, \textit{strategii czystych i mieszanych}, postaci \textit{strategicznej}
oraz \textit{ekstensywnej} gier. Dział ten opisuje również algorytmy służące do obliczenia równowag w rozpatrywanych
przypadkach gier;
\item w dziale \textbf{Implementacja} opisałem szczegóły realizacji programu, strukturę wewnętrzną oraz przyjęty model
danych, jak również algorytmy nie związane bezpośrednio z teorią gier ale wymagane do poprawnego działania programu;
\item w dziale \textbf{Wnioski} opisałem rezultaty otrzymane podczas wdrażania projektu oraz możliwości dalszego
rozwoju powstałego programu;
\item w załącznikach zawarłem informacje na temat kompilacji i użytkowania projektu.
\end{enumerate}

Przy pisaniu pracy wykorzystałem książki \cite{FT-GT}, \cite{O-GT} oraz artykuły \cite{LH-NE}, \cite{PNS-NE}, \cite{SCG-NE}, \cite{SCARF-NR} jako źródła wiedzy teoretycznej na temat rozważanych problemów, a także dokumentacje techniczne wykorzystanych bibliotek podczas rozwijania części praktycznej.

Większość celów pracy można uznać za zrealizowane. Jej wynikiem jest program, który wczytuje definicję gry od użytkownika i znajduje dla niej równowagę Nasha.

\end{document}
