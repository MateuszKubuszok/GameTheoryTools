\documentclass[polish]{standalone}
\usepackage{thesis}

\begin{document}
\pagestyle{headings}

\section*{Wstęp}

Celem pracy było stworzenie zestawu narzędzi do obliczania równowag Nasha w pewnych rodzajach gier skończonych.
Narzędzia te miałyby umożliwić obliczenie ich po wprowadzeniu danych w zdefiniowanym na ich potrzeby formacie,
a następnie zwróceniu wyników użytkownikowi. Jedynymi obecnie dostępnymi narzędziami służącymi do tego celu jest pakiet
\PROG{Gambit}, posiada on jednak dość ograniczoną dokumentację, a jego kod źródłowy, choć otwarty, nie wydaje się łatwy
w rozwijaniu. Dodatkowo pakiet ten jest rozbity na moduły, przez co każdy problem jest rozwiązywany przy pomocy osobnego
narzędzia - może to stanowić problem dla użytkowników oczekujących jednego programu, którego sposób użycia określają
w trakcie jego uruchomienia.

W takiej sytuacji dobrym pomysłem wydawało się stworzenie alternatywy, nastawionej na możliwość łatwego rozwijania
projektu oraz przyjaznego interfejsu umożliwiającego wykorzystanie programu nawet mniej zaawansowanym użytkownikom. Aby
projekt taki miał rację bytu, powinien implementować pewną minimalną funkcjonalność obrazującą jego potencjalne
możliwości. Stworzenie prototypu pozwoliłoby stwierdzić czy taka filozofia projektu przyniosłaby jakąkolwiek korzyść
użytkownikom.

Projekt ten stanowił także okazję do przyjrzenia się istniejącym algorytmom i zaobserwowaniu różnych sposobów na
rozwiązanie problemu szukania równowag Nasha. Kolejni autorzy próbowali odmiennych podjeść - nie są to kolejne ulepszenia
jednego algorytmu.

\marginnote{TODO: dopisać pozostałych jeśli użyję ich prac}
Przy pisaniu pracy wykorzystano książki \textbf{Game Theory} (\textit{Fudenberg, Tirole}) \cite{FT-GT},
\textbf{Teoria Gier} (\textit{Owen}) \cite{O-GT} oraz artykuły:
\begin{enumerate}
\item \textbf{Equilibrium Points of Bimatrix Games} (\textit{Lemke, Howson})
\cite{LH-NE}
\item \textbf{Simple Search Methods for Finding a Nash Equilibrium} (\textit{Porter, Nudelman, Shoham})
\cite{PNS-NE}
\item \textbf{Mixed-Integer Programming Methods for Finding Nash Equilibria} (\textit{Sandholm, Gilpin, Conitzer})
\cite{SCG-NE}
\item \textbf{The Computation of Equilibrium Prices: An Exposition} (\textit{Scarf})
\cite{SCARF-NR}
\end{enumerate}
jako źródła wiedzy teoretycznej na temat rozważanych problemów.

W kolejnych rozdziałach opisałem szczegóły na temat podstaw teoretycznych oraz realizacji projektu:
\begin{enumerate}
\item w dziale \textbf{Teoria} opisano podstawy teoretyczne stojące za projektem. Wprowadzają one pojęcia \textit{gier},
\textit{równowag Nasha}, \textit{strategii czystych i mieszanych}, postaci \textit{strategicznej}
oraz \textit{ekstensywnej} gier. Opisuje również algorytmy służące do obliczenia równowag w rozpatrywanych przypadkach
gier,
\item w dziale \textbf{Implementacja} opisano szczegóły realizacji programu, strukturę wewnętrzną oraz przyjęty model
danych, jak również algorytmy nie związane bezpośrednio z teorią gier ale wymagane do poprawnego działania programu,
\item w dziale \textbf{Wnioski} opisano rezultaty otrzymane podczas wdrażania projektu oraz możliwości dalszego
rozwoju powstałego programu,
\item załączniki zawierają informacje na temat kompilacji i użytkowania projektu.
\end{enumerate}

W trakcie pisania pracy stworzony został prototyp demonstrujący opisywaną ideę. Zdecydowałem się na użycie języka
\LANG{C++}, który umożliwia programowanie z wysoką wydajnością, jak również utrzymanie czytelności kodu. W przypadku
algorytmów wykorzystujących programowanie liniowe, skorzystano z funkcji biblioteki \LIB{GLPK}. Prototyp umożliwia
wczytanie danych przy pomocy parsera, obliczenie rozwiązania zadanego problemu i wyświetlenie wyniku. Posiada
zaimplementowane algorytmy obliczające równowagi Nasha:
\begin{itemize}
\item w grach w postaci strategicznej dla strategii czystych,
\item w grach jednomacierzowych dla strategii mieszanych,
\item w grach dwumacierzowych dla strategii mieszanych,
\item w grach ekstensywnych z informacją doskonałą.
\end{itemize} 

W ogólności cele pracy można uznać za zrealizowane. Jej wynikiem jest działający program, który posiada wymaganą
funkcjonalność co dowodzi realizowalności przyjętych założeń. Ograniczenia czasowe sprawiły jednak, że nie wszystkie
rozpatrywane rozwiązania udało się zaimplementować. Szczegóły zostały uważniej omówione w poszczególnych działach.

\end{document}
